\chapter{Introduction}
\label{cha:introduction}

% Najnowsze odkrycia w dziedzinie technologi informacyjnych umozliwily ludzkosci wytwarzanie gigantycznych strumieni danych. Szacunkowa ilosc danych generowanych na swiecie kazdego dnia przekracza mozliwosci skladowania nawet najpotezniejszych komputerow. Doprowadzilo to do powstania nowych galezi nauki poswieconych przetwarzaniu oraz automatyzacji wnioskowania na podstawie danych (data mining oraz machine learning).

% Ekstrakcja informacji jest procesem wieloetapowym ktory pozwala na kondensacje wiedzy, 

% Problemem samym w sobie stala sie jakosc danych, powszechnym zjawiskiem jest zbieranie oraz skladowanie danych nadmiarowych ktore nastepnie nie uczestnicza w procesie ekstrakcji informacji (wnioskowania). Istnieje cala klasa algorytmow poswiecona wylacznie odkryciu istotnych cech.
% Problem wnioskowania 

Growth of internet - understood as a bigger data link capacity, machines computational power and amount of devices connected - brought into this world a new science branch, the data science. People learned to generate data streams beyond storage capability of single disk, term big data was coined out. Strict definition is not available but rule of thumb states that when dataset to process does not fit in biggest of hard disk. Such amounts provided set of new problems with storage, persistence and availability.
Raw data gathered from various means has to be processed before revealing informations about real world. In case of no knowledge about expected result beforehand this task is referred to as data mining. This task is complex, requires vast amounts of computing power as well as availability to select interesting features of observed data. 

% How cloud computing plays into that

\section{Objectives}
This paper aims to evaluate usefulness of Apache Spark framework in biclustering problem. Steps to achieve this goal include examination of state of knowledge in this field, selection of feasible algorithms, constituting of verification framework, implementation and evaluation in distributed environment.

\section{Structure}
Paper is organised as follows
\begin{enumerate}
\item First chapter contains introduction into the topic and objectives set for the paper.
\item Second chapter describes Biclustering from theoretical perspective with background in using distributed platforms.
\item Third chapter focuses on Spark platform with its applications in distributed computing and data mining tasks.
\end{enumerate}
