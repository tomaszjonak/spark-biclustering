\chapter{Biclustering - overview}
\label{cha:biclustering}
Biclustering is a very specific topic in wider category of data science - data mining. This chapter aims to place biclustering in wider context of automated information retrieval.

\section{Data Science}
Modern quantitative approach to analyze real world phenomena using statistics, data analysis and related methods. Real world phenomenons are described by data - set of samples each described by measured features \. Data can be treated as set of points in n-dimensional space, commonly represented as matrix. Research in data science is split into three main phases: design for data, collection of data, analysis of data. \cite[Ch.\ 2]{Hayashi1998}

Design - selection of features to be observed during collection phase.
Collection - Preforming measurements by various means. Most diverse phase. Depending on scope of research could require experiments on DNA/RNA, webcrawling, monitoring actions on social sites, logging computer infrastructure performance under certain events. List is by no means exhaustive.
Analysis - applying various techniques to obtain structure of data. As example different classes of samples may emerge.

which is set of observations (samples, objects) with set of features
Data in this case can be described as set of observations (objects) described by their features. 

\section{Data mining}



\section{Definitions}
\textbf{Data mining} - automatic or semi-automatic extraction of patterns and relationships, in data. Interdisciplinary subfield of computer science. Data mining tasks include anomaly detection, dependency modeling, clustering, classification, regression and summarization.

\textbf{Clustering} - problem of grouping objects in such a way that objects are more similar to others in group than to others. Partitioning set into multiple subsets given some measure of similarity based on properties (features) of objects.
As example we can take grouping Iris flower data samples into three species, given petal/sepal width/length as features of each sample. 

\textbf{Biclustering} - subfield of cluster analysis. Problem of identifying biclusters - similar behavior of data across subsets of rows and columns possibly different for each bicluster. 
Formulated by J.A Hartigan in 1972 \cite{hartigan1972direct}. Term was first used by Mirkin in 1996 \cite{mirkin1998mathematical}, it is also known as co-clustering, subspace clustering, block clustering and two-mode clustering.

